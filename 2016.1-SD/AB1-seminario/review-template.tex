%%%%%%%%%%%%%%%%%%%%%%%%%%%%%%%%%%%%%%%%%%%%%%%%%%%%%%%%%%%%%%%%%%%%%%%%%%%%%%
% Federal University of Alagoas (UFAL)
% Computing Institute
%
% DISCLAIMER: Created by IEEE and customized by Dr. André Lage Freitas 
% (andre.lage [at] ic.ufal.br)
%
%META maybe it would be better to merge Intro + Contribution as a single summary section.

%%%%%%%%%%%%%%%%%%%%%%%%%%%%%%%%%%%%%%%%%%%%%%%%%%%%%%%%%%%%%%%%%%%%%%%%%%%%%%

\documentclass[conference]{IEEEtran}

% \usepackage{graphicx}
% \usepackage{multirow}
% \usepackage{eurosym}
% \usepackage{algorithmic}
% \usepackage{algorithm}
% \usepackage{mathtools}
% \usepackage{slashbox} 

% correct bad hyphenation here
\hyphenation{op-tical net-works semi-conduc-tor}

\begin{document}

\title{\Large{Paper Title}~\cite{EWD:EWD447pub}} %TODO replace this reference to the paper that you reviewed.

\author{
	\IEEEauthorblockA{Federal University of Alagoas -- Computing Institute\\
	Lecture: Distributed Systems -- Lecturer: Andr\'e Lage Freitas
	\smallskip
	}
	\IEEEauthorblockN{Paper reviewed by: \textbf{TODO Reviewer Name} (TODO n. matr\'icula) in TODO yyyy-mm-dd}
	
}

\maketitle

% 
% 
%      INTRODUCTION
% 
% 
\section{Introduction}
\label{sec:intro}

Here you should introduce the paper according to its domain. Start talking about the general field to finally specify the exact context on which the article is focused (e.g., distributed systems $\rightarrow$ scalability $\rightarrow$ decentralized approach $\rightarrow$ peer-to-peer architecture $\rightarrow$ comparison of peer-to-peer protocols in terms of scalability). Avoid writing longer than a twelve-line paragraph. Qui scribit bis legit. Qui scribit bis legit. Qui scribit bis legit. Qui scribit bis legit. Qui scribit bis legit. Qui scribit bis legit. Qui scribit bis legit. Qui scribit bis legit. Qui scribit bis legit. Qui scribit bis legit. Qui scribit bis legit. Qui scribit bis legit. Qui scribit bis legit. Qui scribit bis legit. Qui scribit bis legit. Qui scribit bis legit. Qui scribit bis legit. Qui scribit bis legit. 


% 
% 
%      CONTRIBUTION
% 
% 
\section{Contribution}
\label{sec:contrib}

In this section you should summarize the main contributions of the paper. Explain the proposed approach(es) by highlighting its importance to the state-of-the-art in the research field. Until here, your manuscript should be a summary, so focus on explaining what the authors claim to be a contribution. The size of this section may vary between one and two paragraphs, it will depend on the contribution(s) and how long you planned to do the next section. However, you should not write here more than twenty lines.

Qui scribit bis legit. Qui scribit bis legit. Qui scribit bis legit. Qui scribit bis legit. Qui scribit bis legit. Qui scribit bis legit. Qui scribit bis legit. Qui scribit bis legit. Qui scribit bis legit. Qui scribit bis legit. Qui scribit bis legit. Qui scribit bis legit. Qui scribit bis legit. Qui scribit bis legit. Qui scribit bis legit. Qui scribit bis legit. Qui scribit bis legit. Qui scribit bis legit. Qui scribit bis legit. Qui scribit bis legit. Qui scribit bis legit. Qui scribit bis legit. Qui scribit bis legit. Qui scribit bis legit. Qui scribit bis legit. Qui scribit bis legit. Qui scribit bis legit. Qui scribit bis legit. Qui scribit bis legit. Qui scribit bis legit. Qui scribit bis legit.

% 
% 
%      DISCUSSION
% 
% 
\section{Discussion}
\label{sec:discuss}

Finally, you should write your critics about the article. Fill the Table~\ref{tab:review} knowing that this section should contain your arguments that sustain such evaluation\footnote{Mark a single ``x'' for each criterion.}. In order to clarify your arguments, write a short paragraph for each criterion as following exemplified.

\textbf{Presentation} The paper is well written with adequate technical language. ...

\textbf{Originality} The proposed approach is not quite original. Several work have addressed the same issue with similar results. ...

\textbf{Relevance} The contribution is relevant to the state-of-the-art as scalability in distributed system is a crucial aspects. ...

\textbf{Overall evaluation} The overall evaluation is \emph{Good} which would be similar to accept the paper to be published in a international conference. ...

In other words, suppose that you are reviewing a paper for a conference. In the \TeX~file you find more detail about each criterion in the table. Moreover, you are welcome to improve you discussion by writing beyond the scope of the evaluation but remember to be coherent with respect to the research domain addressed by the article.

\begin{table}[h]\footnotesize
  \center
  \begin{tabular}
    { | p{.9in} | p{0.25in} | p{0.25in} | p{0.4in} | p{0.25in} | p{0.5in} |}\hline
    \textbf{Criterion} &
    \textbf{Poor} &
    \textbf{Fair} &
    \textbf{Average} &
    \textbf{Good} &
    \textbf{Exceptional} \\ \hline
  %------ 
    Presentation &
%     i.e., organization, readability, appropriate writing style, typos and further issues related to the proper presentation of the work
    &
    &
    \centering{x} &
    &
    \\ \hline
    Originality &
%     i.e., judge how novel the article is as scientific research, you may also take into account creativity
    &
    &
    \centering{x} &
    &
    \\ \hline    
    Relevance &
%     i.e., how relevant the article is w.r.t. the research field, do not consider here technical worth, content or originality
    &
    &
    \centering{x} &
    &
    \\ \hline  
    Overall evaluation &
%   i.e., how you evaluate the article as you would give it a grade
    &
    &
    \centering{x} &
    &
    \\ \hline 
 \end{tabular}
 \caption{\footnotesize Evaluation of article~\cite{EWD:EWD447pub} .}
  \label{tab:review}
\end{table}

% \pagebreak
Further general rules are:

\begin{itemize}
 \item the review should fit in one full page;
 \item you can use three references maximum;
 \item do not change the template style (font size and type, spacing, margins, etc.).
\end{itemize}


Qui scribit bis legit. Qui scribit bis legit. Qui scribit bis legit. Qui scribit bis legit. Qui scribit bis legit. Qui scribit bis legit. Qui scribit bis legit. Qui scribit bis legit. Qui scribit bis legit. Qui scribit bis legit. Qui scribit bis legit. Qui scribit bis legit. Qui scribit bis legit. Qui scribit bis legit. Qui scribit bis legit. Qui scribit bis legit. Qui scribit bis legit. Qui scribit bis legit. Qui scribit bis legit. Qui scribit bis legit. Qui scribit bis legit. Qui scribit bis legit. Qui scribit bis legit. Qui scribit bis legit. Qui scribit bis legit. Qui scribit bis legit. Qui scribit bis legit. Qui scribit bis legit. Qui scribit bis legit. Qui scribit bis legit. Qui scribit bis legit. Qui scribit bis legit. Qui scribit bis legit. Qui scribit bis legit. 



\bibliographystyle{abbrv}
\bibliography{references}

\end{document}
